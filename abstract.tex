
State of the art visual 3D reconstruction techniques can provide rich ecological and structural information from underwater imagery. When coupled with underwater robotics, the additional navigation suite helps to georeference the reconstructions as well as allow for systematic coverage of areas much larger than the footprint of a single image. However, access to suitable robotic systems is limited and requires specialised operators, which reduces the widespread use of these techniques. Furthermore, field conditions are often poor, especially in shallow habitats, and human control of equipment is required; however, humans do not have the capacity to navigate precisely over large extents to produce consistent image overlap.  We present a simple method to quickly generate structured, repeatable and large extent surveys ($\sim{100m^{2}}$ in $\sim{15 min}$) that can be performed with one swimmer and one support person. The amount of image overlap is a design parameter, allowing for surveys that can then be used in an automated processing pipeline to generate georeferenced 3D reconstructions, orthographically projected mosaics and structural complexity indices. The individual georeferenced images or full mosaics can also be labeled for benthic diversity and cover estimates. We describe the method and present results from a coral reef monitoring program around Lizard Island in the Great Barrier Reef, Australia.
  
  