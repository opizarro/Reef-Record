
State of the art visual 3D reconstruction techniques can provide rich structural information from underwater imagery. When coupled with underwater robotics, the additional navigation suite helps to georeference the reconstructions as well as allow for systematic coverage of areas much larger than the footprint of a single image. However, access to suitable robotic systems is limited and requires specialised operators, which curtails the use and impact of these techniques.
We present a simple method to quickly generate repeatable, high resolution (\sim{1mm^{2}}) and large extent (\sim{100m^{2}}) surveys that can be performed with two swimmers (\sim{15 min} for {100m^{2}}). The survey data can then be used in an automated processing pipeline to generate georeferenced 3D reconstructions, orthographically projected mosaics and structural complexity indeces. The individual georeferenced images or full mosaics can also be labeled for benthic cover estimates.
We describe the method and present typical results from coral reef surveys around Lizard Island, QLD and temperate rocky reefs in Botany bay, NSW. We characterise the quality of data products and repeatability of the method, as well as discuss future enhancements.
  
  