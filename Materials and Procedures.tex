\section{Materials and Procedures}
%Provides a detailed set of instructions for implementing the method, including all materials required and all procedures followed. Describes methods used in evaluation and intercalibration or intercomparison studies. Presents data sources, data extraction protocols and analytical methods used in reviews and metaanalyses.

%Whenever possible and appropriate, manuscripts must present complete instructions for the recommended procedure, analogous to a good cookbook or an easy-to-use laboratory manual. Descriptions of equipment and apparatus must provide a similar level of detail regarding the contruction and operation of the device. Evaluation and intercalibration or intercomparison studies may refer to published descriptions of existing methods, but should describe in reproducible detail how the present study was conducted. Metaanalytical studies should provide details regarding data sources and extraction, and analytical methods used.

For down-looking cameras, systematic surveys covering areas much larger than the footprint of a single image require multiple views of the same scene points (i.e. `image overlap') in order to relate the multiple images into a composite representation such as a 3D reconstruction or an orthographic mosaic. Overlap along the direction of motion depends on the field of view, altitude and motion between image capture instants. Equation for min number of views here. Solve for speed.
Overlap across tracks depends on the field of view, altitude and spacing between tracks. 
Equation for across track overlap. Solve for spacing.
Field of view can be estimated from a camera calibration or, more approximately, by using the effective focal length in water and the imaging chip size. Our configuration currently has a horizontal (across track) field of view of 42deg, and 34deg along track. At a desired altitude of 2m, and a minimum of four views of each scene points, we have forward speed X and trackline spacing Y. 
% curves of overlap?


For a drum of radius $R$ and angle $\alpha$ along the circumference of the drum, the tip of the line is located at polar coordinates $r$, $\phi$:  
\begin{equation}
r = R\cdot\sqrt{1+\alpha^2} \\
\phi = \alpha - \arctan{\alpha} 
\end{equation}
The outer boundary of the involute, $r$, grows in a near linear fashion with the number of revolutions, while the path (or arc) length, $L$, is given by
\begin{equation}
L=\frac{R}{2}\cdot\alpha^2,
\end{equation}
growing quadratically with the number of revolutions. 
From a survey design point of view, the choice of drum diameter and length of line, $M$, determine the number of revolutions needed to unwind the line as well as the path length. Together with the swimming speed this determines the duration. The final radius $r_M$ is given by the hyptothenus of the right angle triangle formed by the drum radius and the line extended at right angles, $r_M = \sqrt{M^2 + R^2}$ and therefore the total angle swept to unwind the whole drum is 
\begin{equation}
\alpha_M = \sqrt{\left(\frac{r_M}{R}\right)^2 - 1}
\end{equation}
substituting for $r_M$ yields $\alpha_M = \frac{M}{R}$ and therefore $L_M = \frac{M^2}{2R}$. The time to complete a survey, $t_M$, at a swimming speed $s$ is then
\begin{equation}
t_M = \frac{L_M}{s} = \frac{M^2}{2R\cdot s}
\end{equation}
For a line 6m long, a drum of 0.15m in diameter and a swimming speed of 0.2 $m/s$, the path length of the spiral pattern is X m and will take Y seconds.


    
  
  
  
  
  