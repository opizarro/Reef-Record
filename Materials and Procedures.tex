
% SBW: This assumes that there is no overlap in the line when wound around the drum.  Not sure if it's worth mentioning this as an assumption.
For a drum of radius $R$ and angle $\alpha$ along the circumference of the drum, the tip of the line is located at polar coordinates $r$, $\phi$:  
\begin{equation}
r = R\cdot\sqrt{1+\alpha^2} \\
\phi = \alpha - \arctan{\alpha} 
\end{equation}
The outer boundary of the involute, $r$, grows in a near linear fashion with the number of revolutions, while the path (or arc) length, $L$, is given by
\begin{equation}
L=\frac{R}{2}\cdot\alpha^2,
\end{equation}
growing quadratically with the number of revolutions. 
From a survey design point of view, the choice of drum diameter and length of line, $M$, determine the number of revolutions needed to unwind the line as well as the path length. Together with the swimming speed this determines the duration of a deployment. The final radius $r_M$ is given by the hypotenuse of the right angle triangle formed by the drum radius and the line extended at right angles, $r_M = \sqrt{M^2 + R^2}$ and therefore the total angle swept to unwind the whole drum is 
\begin{equation}
\alpha_M = \sqrt{\left(\frac{r_M}{R}\right)^2 - 1}
\end{equation}
substituting for $r_M$ yields $\alpha_M = \frac{M}{R}$ and therefore $L_M = \frac{M^2}{2R}$. The time to complete a survey, $t_M$, at a swimming speed $s$ is then
\begin{equation}
t_M = \frac{L_M}{s} = \frac{M^2}{2R\cdot s}
\end{equation}
For a line 6m long, a drum of 0.16m in diameter, the path length of the spiral pattern is 225 m. At a swimming speed of 0.3 m/s, the spiral pattern would take 750 s to execute.


    
  
  
  
  
  