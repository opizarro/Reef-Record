\section{Materials and Procedures}
%Provides a detailed set of instructions for implementing the method, including all materials required and all procedures followed. Describes methods used in evaluation and intercalibration or intercomparison studies. Presents data sources, data extraction protocols and analytical methods used in reviews and metaanalyses.

%Whenever possible and appropriate, manuscripts must present complete instructions for the recommended procedure, analogous to a good cookbook or an easy-to-use laboratory manual. Descriptions of equipment and apparatus must provide a similar level of detail regarding the contruction and operation of the device. Evaluation and intercalibration or intercomparison studies may refer to published descriptions of existing methods, but should describe in reproducible detail how the present study was conducted. Metaanalytical studies should provide details regarding data sources and extraction, and analytical methods used.
\subsection{Materials}
%"Materials" includes expendible and non-expendible supplies, equipment, and solutions. Materials should be listed as completely as possible, with careful attention to providing enough information to assist new users in adopting the method. Descriptions of needed equipment must note the essential features of the equipment in sufficient detail to allow users to obtain similar devices, or authors may refer to specific commercial products in lieu of a detailed description. The composition of any solutions must be stated explicitly (e.g. grams per liter of a specific compound).
We rely on three three major components for the data acquisition
\begin{enumerate}
\item stereo imaging pacakge. An updated version of the one used in \cite{Henderson_2013}
\item drum and line
\item star picket or base
\end{enumerate}

in addition the imagery and navigation information are processed using the ACFR pipeline \cite{Johnson_Roberson_2010} \cite{Mahon_2008} \cite{Johnson_Roberson_2013} to generate georeferenced estimates of the stereo imaging package trajectory and the 3D observed by the cameras, as a textured mesh. Multi-scale structural complexity indeces such as rugosity are then derived from those meshes \cite{Friedman_2012} 

\subsection{Procedures}
%A stepwise description of all procedures used, divided into sub-procedures as necessary for clarity. Sufficient detail must be provided that a reader may readily adopt and employ the method as described, at least under the conditions described.
%Limnology and Oceanography: Methods will not accept manuscripts that describe laboratory and field techniques, equipment, analyses, and other methods, in insufficient detail to be reproduced by others.