\section{Materials and Procedures}
%Provides a detailed set of instructions for implementing the method, including all materials required and all procedures followed. Describes methods used in evaluation and intercalibration or intercomparison studies. Presents data sources, data extraction protocols and analytical methods used in reviews and metaanalyses.

%Whenever possible and appropriate, manuscripts must present complete instructions for the recommended procedure, analogous to a good cookbook or an easy-to-use laboratory manual. Descriptions of equipment and apparatus must provide a similar level of detail regarding the contruction and operation of the device. Evaluation and intercalibration or intercomparison studies may refer to published descriptions of existing methods, but should describe in reproducible detail how the present study was conducted. Metaanalytical studies should provide details regarding data sources and extraction, and analytical methods used.
Our approach for systematic, full coverage underwater photographic surveys is to use a line wound around a drum as a guide. Unwinding the line under tension constrains motion as to generate spiral pattern. The curve traced by the tip of the line corresponds to the involute of a circle (i.e., the cross section of the drum), with constant separation distance between revolutions corresponding to the perimeter of the circle. 

Archimidean spirals, that resemble involutes of a circle, have been investigated in the reconstruction of surfaces in metrology \cite{Wieczorowski_2001} and in estimating patchy distributions \cite{Kalikhman_2006} though these uses are concerned with sparse sampling of the area of interest, while our focus is in an operationally simple way to achieve full photographic coverage of an area with limited field of view (and footprint) imaging. Our approach is practical for full coverage of \sim{100m^{2}} areas.

For a drum of radius $R$ and angle $\alpha$ along the circumference of the drum, the tip of the line is located at polar coordinates $r$, $\phi$:  
\begin{equation}
r = R\cdot\sqrt{1+\alpha^2} \\
\phi = \alpha - \arctan{\alpha} 
\end{equation}
The outer boundary of the involute, $r$, grows in a near linear fashion with the number of revolutions, while the path (or arc) length, $L$, is given by
\begin{equation}
L=\frac{R}{2}\cdot\alpha^2,
\end{equation}
growing quadratically with the number of revolutions. The time to complete a survey, $t$, at a swimming speed $s$ is then
\begin{equation}
t = \frac{L}{s} = \frac{R}{2s}\cdot\alpha^2
\end{equation}
From a survey design point of view, the choice of drum diameter and length of line, $M$, determine the number of revolutions needed to unwind the line as well as the path length. Together with the swimming speed this determines the duration. The final radius $r_M$ is given by the hyptothenus of the right angle triangle formed by the drum radius and the line extended at right angles, $r_M = \sqrt{M^2 + R^2}$ and therefore the total angle swept to unwind the whole drum is 
\begin{equation}
\alpha_M = \sqrt{\lgroup\frac{r_M}{R}\rgroup^2 - 1}
\end{equation}

\subsection{Materials}
%"Materials" includes expendible and non-expendible supplies, equipment, and solutions. Materials should be listed as completely as possible, with careful attention to providing enough information to assist new users in adopting the method. Descriptions of needed equipment must note the essential features of the equipment in sufficient detail to allow users to obtain similar devices, or authors may refer to specific commercial products in lieu of a detailed description. The composition of any solutions must be stated explicitly (e.g. grams per liter of a specific compound).
We rely on three three major components for the data acquisition
\begin{enumerate}
\item stereo imaging package. An updated version of the one used in \cite{Henderson_2013} \cite{Camilli_2007}. We require properly calibrated stereo pair and sensor suite, as well as the pose of each sensor relative to the navigation center \cite{Johnson_Roberson_2013} \cite{}.
\item drum and line. Typically a series of reef records are collect. Prior to starting, it is important to select the diameter of the drum and lenght of the line to determine the size of the patch and the image overlap from the spiral spacing. The resulting pattern will have the form shown in Figure \ref{fig:spiral}.

% photo of drum and line

% full equation and approximate equation for drum and line length


\item star picket or base
% photo of base / star picket

\end{enumerate}



in addition the imagery and navigation information are processed using the ACFR pipeline \cite{Johnson_Roberson_2010} \cite{Mahon_2008} \cite{Johnson_Roberson_2013} to generate georeferenced estimates of the stereo imaging package trajectory and the 3D observed by the cameras, as a textured mesh. Multi-scale structural complexity indeces such as rugosity are then derived from those meshes \cite{Friedman_2012} 

\subsection{Procedures}
%A stepwise description of all procedures used, divided into sub-procedures as necessary for clarity. Sufficient detail must be provided that a reader may readily adopt and employ the method as described, at least under the conditions described.
%Limnology and Oceanography: Methods will not accept manuscripts that describe laboratory and field techniques, equipment, analyses, and other methods, in insufficient detail to be reproduced by others.

In the field\begin{enumerate}
\item Select location, such that the spiral covers the area of interest.
\item Drive star picket or install central base
\item attach pole and drum to base
\item clip imaging package to line
\item swim forward while keeping tension on the line and desired altitude.
\item detach line and swim back over the pole, completing at least for `spokes'
\end{enumerate}

The imagery and navigation logs are post processed to estimate camera poses and 3D composite meshes of the surveyed area. Rugosity, slope and aspect can me calculated from the triangulated mesh at multiple scales and resultions. 

