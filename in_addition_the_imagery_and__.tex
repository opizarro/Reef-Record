
in addition the imagery and navigation information are processed using the ACFR pipeline \cite{Johnson_Roberson_2010} \cite{Mahon_2008} \cite{Johnson_Roberson_2013} to generate georeferenced estimates of the stereo imaging package trajectory and the 3D observed by the cameras, as a textured mesh. Multi-scale structural complexity indeces such as rugosity are then derived from those meshes \cite{Friedman_2012} 

\subsection{Procedures}
%A stepwise description of all procedures used, divided into sub-procedures as necessary for clarity. Sufficient detail must be provided that a reader may readily adopt and employ the method as described, at least under the conditions described.
%Limnology and Oceanography: Methods will not accept manuscripts that describe laboratory and field techniques, equipment, analyses, and other methods, in insufficient detail to be reproduced by others.

In the field\begin{enumerate}
\item Select location, such that the spiral covers the area of interest.
\item Drive star picket or install central base
\item attach pole and drum to base
\item clip imaging package to line
\item swim forward while keeping tension on the line and desired altitude.
\item detach line and swim back over the pole, completing at least for `spokes'
\end{enumerate}

The imagery and navigation logs are post processed to estimate camera poses and 3D composite meshes of the surveyed area. Rugosity, slope and aspect can me calculated from the triangulated mesh at multiple scales and resultions. 


    
  