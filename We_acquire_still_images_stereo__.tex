We acquire still images (stereo pairs) at 2 Hz, providing ample overlap along track. Figure~\ref{fig:real_spiral} shows an overview of one of the spirals. %  The black dots represent the estimate location of the camera throughout the survey.  The red lines join camera locations for which image features have been matched, representing effective image overlap.  The green line represents the GPS fixes collected throughout the survey.  
These observations are fused with depth and orientation information in addition to the imagery to produce a georeferenced estimate of the path followed by the platform.

Based on the 35 surveys from the May 2015 expedition, the average, standard deviation, maximum and minimum for survey duration, number of stereo pairs, and image matches are summarised in table \ref{tab:SurveyStats}. The variation in time and image overlap is attributable to the actual altitude and path followed on each survey. In the case of snorkeling on reef flats, tidal variations affect water depth which determines the imaging altitude and image footprint ( see Sec.~\ref{sec:MandP} and Fig.~\ref{fig:fov} ). Swell and currents act as disturbances while swimming the spirals that affect swimming speed and the actual path followed (i.e. the line constrains motion away from the central pole but not towards it). Given the average number of images and matched images, each image overlaps with approximately 10.4 other ones, however their distribution is not uniform given the tight turn radius towards the centre of the survey. Figure ~\ref{fig:LCdensity} illustrates the density of image matches for an example survey. The large central values are consistent with a camera initially mostly rotating around a nearby axis. 

A secondary factor that can affect the number of image matches is the presence of caustics from the lensing effect of surface waves and ripples. In shallow water these are noticeable as fast moving bands of bright light. The effect is less pronounced early morning and late afternoon, when the sun is closer to the horizon. In general, a successful 3D reconstruction depends more on having enough water depth under the cameras than the position of the sun. However, strong banding will be apparent in the textured reconstruction and mosaics. Techniques that rely on the redundancy of video rates can significantly reduce these artifacts \cite{Gracias_2008}.

%For our setup this resulted in a network of camera poses with approximately 1800 image pairs and 19000 links (relating two image pairs), each stereo pair overlaps approximately with 21 other stereo pairs, making for a well-connected photogrammetric network.

