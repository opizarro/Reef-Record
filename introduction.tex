\section{Introduction} 
% Establishes the need for a new or improved method and introduces the method in concept. Discusses problems with existing methods that will be evaluated. Establishes need for intercomparison or intercalibration studies. Discusses the current status of a field and establishes the need for a critical review or metaanalytical study.
%traditional benthic surveys with divers pros and cons
%advanced surveys with AUVs pros and cons

%reconstruction and rugosity

%monitoring
%labelling
Effective techniques to quantify underwater benthic community composition and habitat structure are of importance to ecologists and resource managers. Species diversity and abundance data are used to gage the makeup and trajectories of communities (Sweatman: AIMS) as well as responses to disturbances, ranging from shorter-term storms and thermal events (ref) to longer-term events associate with climate change \cite{pending}. The structural complexity built by benthic communities creates space for associated organisms to live. Benthic habitat with higher levels of structural complexity commonly contain more associated species (e.g., fishes and crustaceans) \cite{pending} and tend to bounce back faster following disturbances (Graham: Predicting climate-driven regime shifts versus rebound potential in coral reefs), and subsequently habitat complexity is often a good indicator of the health and functioning of ecosystems \cite{pending}. Enabling fast and reliable observation of benthic community composition and structural complexity over large areas in difficult field conditions will greatly improve tests of ecological theory and the effectiveness of monitoring programs.

%Rugosity, is a common index for structural complexity. It can be thought of as a the ratio between
%defined as the ratio between the `drapped' length of a bottom profile and the shortest distance between the profile endpoints.
\newline


Traditional techniques to estimate community composition and structural complexity are typically labour intensive, 1-dimensional, and capture data at small scales. For example, to capture community composition, field ecologists tend to use the line intercept transect technique, which captures the proportion of a line that overlaps different benthic categories, such as algae, sponge and coral (Loya ***). Simiarly, for structural complexity, field ecologists tend to use the chain method \cite{Luckhurst_1978} \cite{Friedlander_1998}, which measures the length of chain draped closely over the benthos required to span a distance between two points. The limited scale of such techniques means that it is necessary for high levels of replication in order to accurately capture benthic community composition and habitat structure. Furthermore, to reduce labour and create a general overview of benthic characteristics for comparison within and among sites, transects and chains are set randomly, and subsequently spatially explicit detail cannot be recreated. 

\newline

Recent advances in robotics and computer vision enable 3D reconstructions of bathymetry from which multi-scale structural complexity can be estimated quickly and reliably \cite{Friedman_2012}. These techniques rely on combining overlapping images into a composite 3D reconstruction, and while they can scale to areas of tens to thousands of square meters consisting of tens of thousands of images, they need a systematic way of covering the survey site. This is an ideal task for a properly instrumented underwater robot, which can carry down-looking cameras and be preprogrammed to follow a survey pattern to collect the desired imagery \cite{Williams_2012}. However, the use of robots is still logistically complex, requiring specialised personnel in the field to operate and service the machines. 
This paper presents a simple, repeatable and low cost alternative to generate systematic surveys for visual 3D reconstructions of benthic habitats. It removes the need for high end navigation and controls and relies instead on constraining motion of a swimmer carrying the imaging equipment. 

\subsection{Visual reconstructions}
Structure from Motion (SFM) techniques \cite{Hartley_2004} can estimate the 3D structure in a scene from imagery collected from multiple viewpoints in a largely automatic fashion. While its fundamentals are shared with traditional photogrammetry \cite{Jones_1982}, in its simplest form SFM uses a single moving camera, which results in a loss of scale of the reconstruction, as well as the global position and orientation of the reconstruction. By including reference objects in the imagery scale can be estimated. Surveyed markers can also provide scale and georeferencing though they require additional infrastructure and time to deploy. In many instances, these markers are only used to scale and register the visual reconstruction after the fact rather than constraining the solution as additional observations.
Other approaches such as Simultaneous Localisation and Mapping (SLAM) \cite{Thrun_2008} can fuse multiple sources of camera position (e.g., GPS on the surface, depth, attitude) and scene structure to yield a robust georeferenced navigation solutions.
%While there are examples diver-based visual SLAM \cite{Henderson_2013} and using off-the-shelf SFM packages \cite{Burns_2015}, the ability to perform systematic surveys has been limited and cumbersome.

\subsection{Systematic surveys}
Given the strong attenuation of light underwater \cite{Duntley_1963}, optical imaging is typically performed at distances of around 0.5-3m off the bottom. This limits the footprint of images to be only a few square meters, which implies that extents of tens to thousands of square meters require handling tens to thousands of images. In addition, achieving full coverage is challenging underwater. Systematically covering an area without leaving gaps or `holes' and providing adequate overlap has usually been accomplished by robots that follow a `mow the lawn' pattern at a near-constant altitude off the bottom \cite{Bingham_2010}\cite{Williams_2010}, ensuring image overlap across parallel tracklines. These robots carry a sophisticated navigation suite that typically measures velocity with an acoustic Doppler Velocity Log (DVL), depth through pressure, precise attitude sensing with an inertial measurement unit, and positioning with acoustic transponders, as well as a combination of thrusters and control surfaces to follow a desired survey pattern. Their operating complexity and costs put them beyond many day-to-day scientific applications that would benefit from large scale visual reconstructions.
Diver-held imaging packages can also deliver broad area coverage though the replication of systematic `mow the lawn' patterns with humans requires ropes as visual guides and additional people in the water to handle the lines \cite{Henderson_2013}. Others \cite{Burns_2015} have relied on the diver to swim an approximate grid pattern unaided by external guides but this approach does not scale well to larger areas, tends to break down for narrow line spacing, or in the presence of disturbances such as currents or swell. It is also a tedious task that depends heavily on the skill of the diver in keeping track of their progress.

Our approach for systematic, full coverage underwater photographic surveys is to use a line wound around a drum as a guide. Unwinding the line under tension constrains motion to a spiral pattern. The curve traced by the tip of the line (and the imaging platform attached to it) corresponds to the involute of a circle (i.e., the cross section of the drum), with constant separation distance between revolutions corresponding to the perimeter of the circle. A more cumbersome version of this was attempted in 2007 in the expedition reported in \cite{Camilli_2007} with the line coiled on the imaging platform and the diver having to let out an amount of line equivalent to the trackline spacing after completing each revolution, resulting in concentric circles. More recently, a `minute mosaic' \cite{gintert2012third} uses a rebar pin as a visual reference for a diver to complete three revolutions with increasing radius while collecting imagery for a 2D mosaic. Since it depends on the diver's assessment of distance to the pin, the areas covered varied from 19 to 44 m$^{2}$. This is also likely to result in variable image overlap between revolutions, affecting the quality of the composite and potentially limiting its value for systematic repeat surveys.

Archimidean spirals, that resemble involutes of a circle, have been investigated in the reconstruction of surfaces in metrology \cite{Wieczorowski_2001} and in estimating patchy distributions \cite{Kalikhman_2006} though these uses are concerned with sparse sampling of the area of interest, while our focus is in an operationally simple way to achieve full photographic coverage of an area with limited field of view (and footprint) imaging. Our approach is practical for full coverage of \sim{100m$^{2}$} areas, which corresponds to a radius of \sim{6m}. Much larger lengths may become unwieldy underwater and increase the chances of entanglement and snagging.

%Bi$_{2}$Sr$_{2}$Ca$_{2}$Cu$_{3}$O$_{10 + \delta}$ (Bi-2223). We conclude with a revisitation of the work of  which can also be found at this URL: \url{http://adsabs.harvard.edu/abs%/1975CMaPh..43..199H}.
  