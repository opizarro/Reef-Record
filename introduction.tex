\section{Introduction} 
% Establishes the need for a new or improved method and introduces the method in concept. Discusses problems with existing methods that will be evaluated. Establishes need for intercomparison or intercalibration studies. Discusses the current status of a field and establishes the need for a critical review or metaanalytical study.
%traditional benthic surveys with divers pros and cons
%advanced surveys with AUVs pros and cons

%reconstruction and rugosity

%monitoring
%labelling
Effective techniques to quantitatively describe underwater habitat structure are of interest to ecologist since structural complexity is relevant to species distribution \cite{cite} as well as tied to the health of habitats \cite{cite}. There is also interest in tracking structural complexity through time and relating it to physical and anthropogenic disturbances, as well as changes in composition of communities. 
%Rugosity, is a common index for structural complexity. It can be thought of as a the ratio between
%defined as the ratio between the `drapped' length of a bottom profile and the shortest distance between the profile endpoints.

Traditional techniques to estimate rugosity, such as the chain method \cite{Luckhurst_1978} \cite{Friedlander_1998} and field profile gauges \cite{McCormick_1994}, are labour intensive, damage the environment and yield sparse coverage. Some improvements have partially addressed these concerns but remain sparse \cite{Dustan_2013}.  Recent advances in robotics and computer vision enable 3D reconstructions of bathymetry from which multi-scale structural complexity can be estimated quickly and reliably \cite{Friedman_2012}. These techniques rely on combining overlapping images into a composite reconstruction, and while they can scale to areas of tens or hundreds of meters consisting of thousands of images, they need a systematic way of covering the survey site. This is an ideal task for a properly instrumented underwater robot, which can be preprogrammed to follow a survey pattern to collect the desired imagery \cite{Williams_2012}. However, the use of robots is still logistically complex, requiring specialised personnel in the field to operate and service the machines.
This paper presents a simple, repeatable and low cost alternative to generate systematic surveys for visual 3D reconstructions of benthic habitats. It removes the need for high end navigation and controls and relies instead on constraining motion of a swimmer carrying the imaging equipment. 

\subsection{Visual reconstructions}
Structure from Motion (SFM) techniques can estimate the 3D structure in a scene from imagery collected from multiple viewpoints. In its simplest form SFM uses a single moving camera, which results in a loss of scale of the reconstruction. By including reference objects in the imagery scale can be estimated.
Other approaches such as Simultaneous Localisation and Mapping (SLAM) can fuse multiple sources of camera position (e.g., GPS on the surface, depth, attitude) and scene structure to yield a robust georeferenced solution.

\subsection{Systematic surveys}
Given the strong attenuation of light underwater \cite{DUNTLEY_1963}, optical imaging is typically performed at distances of around 0.5-3m off the bottom. This limits the footprint of images to be only a few square metres, which implies that extents of tens to thousands of square meters require handling tens to thousands of images. In addition, achieving full coverage is challenging underwater. Systematically covering an area without leaving `holes' and providing adequate overlap. 

%Bi$_{2}$Sr$_{2}$Ca$_{2}$Cu$_{3}$O$_{10 + \delta}$ (Bi-2223). We conclude with a revisitation of the work of  which can also be found at this URL: \url{http://adsabs.harvard.edu/abs%/1975CMaPh..43..199H}.