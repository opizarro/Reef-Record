\section{Introduction} 
% Establishes the need for a new or improved method and introduces the method in concept. Discusses problems with existing methods that will be evaluated. Establishes need for intercomparison or intercalibration studies. Discusses the current status of a field and establishes the need for a critical review or metaanalytical study.
%traditional benthic surveys with divers pros and cons
%advanced surveys with AUVs pros and cons
\cite{Williams_2012}
%reconstruction and rugosity
\cite{Friedman_2012} \cite{Dustan_2013}
%monitoring
%labelling
Structural complexity of a habitat is relevant to ecologists studying distribution and diversity of species. There is also interest in tracking structural complexity through time and relating it to physical and anthropogenic disturbances, as well as changes in composition of communities. 
%Rugosity, is a common index for structural complexity. It can be thought of as a the ratio between
%defined as the ratio between the `drapped' length of a bottom profile and the shortest distance between the profile endpoints.

Traditional techniques to estimate rugosity, such as the chain method, are labour intensive, damage the environment and yield sparse coverage. Recent advances in robotics and computer vision enable 3D reconstructions of bathymetry from which multi-scale structural complexity can be estimated quickly and reliably \cite{Friedman_2012}. These techniques rely on combining images into a composite reconstruction, and while they can scale to areas of tens or hundreds of meters consisting of thousands of images, they need a systematic way of covering the survey site. This is an ideal task for an underwater robot, which can be preprogrammed to follow a survey pattern to collect the desired imagery. However, the use of robots is still logistically complex, requiring specialised engineers in the field to operate and service the machines.
This paper presents a simple, low cost alternative to generate systematic surveys for visual 3D reconstructions of benthic habitats. 



%Bi$_{2}$Sr$_{2}$Ca$_{2}$Cu$_{3}$O$_{10 + \delta}$ (Bi-2223). We conclude with a revisitation of the work of  which can also be found at this URL: \url{http://adsabs.harvard.edu/abs%/1975CMaPh..43..199H}.