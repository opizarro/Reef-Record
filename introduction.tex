\section{Introduction} 
% Establishes the need for a new or improved method and introduces the method in concept. Discusses problems with existing methods that will be evaluated. Establishes need for intercomparison or intercalibration studies. Discusses the current status of a field and establishes the need for a critical review or metaanalytical study.
%traditional benthic surveys with divers pros and cons
%advanced surveys with AUVs pros and cons

%reconstruction and rugosity

%monitoring
%labelling
Effective techniques to quantitatively describe underwater habitat structure are of interest to ecologist since structural complexity is relevant to species distribution \cite{pending} as well as tied to the health of habitats \cite{pending}. There is also interest in tracking structural complexity through time and relating it to physical and anthropogenic disturbances, as well as changes in composition of communities. Enabling fast and reliable observation of quantative strucutural complexity indeces opens up the possibility of using them as proxies for related ecological variables.
%Rugosity, is a common index for structural complexity. It can be thought of as a the ratio between
%defined as the ratio between the `drapped' length of a bottom profile and the shortest distance between the profile endpoints.
However, traditional techniques to estimate rugosity, such as the chain method \cite{Luckhurst_1978} \cite{Friedlander_1998} and field profile gauges \cite{McCormick_1994}, are labour intensive, damage the environment and yield sparse coverage. Some improvements have partially addressed these concerns but remain sparse \cite{Dustan_2013}.  Recent advances in robotics and computer vision enable 3D reconstructions of bathymetry from which multi-scale structural complexity can be estimated quickly and reliably \cite{Friedman_2012}. These techniques rely on combining overlapping images into a composite 3D reconstruction, and while they can scale to areas of tens to thousands of square meters consisting of tens of thousands of images, they need a systematic way of covering the survey site. This is an ideal task for a properly instrumented underwater robot, which can carry down-looking cameras and be preprogrammed to follow a survey pattern to collect the desired imagery \cite{Williams_2012}. However, the use of robots is still logistically complex, requiring specialised personnel in the field to operate and service the machines.
This paper presents a simple, repeatable and low cost alternative to generate systematic surveys for visual 3D reconstructions of benthic habitats. It removes the need for high end navigation and controls and relies instead on constraining motion of a swimmer carrying the imaging equipment. 

\subsection{Visual reconstructions}
Structure from Motion (SFM) techniques \cite{Hartley_2004} can estimate the 3D structure in a scene from imagery collected from multiple viewpoints in largely automatic fashion. While its fundamentals are shared with traditional photogrammetry \cite{Jones_1982},   In its simplest form SFM uses a single moving camera, which results in a loss of scale of the reconstruction. By including reference objects in the imagery scale can be estimated.
Other approaches such as Simultaneous Localisation and Mapping (SLAM) \cite{Thrun_2008} can fuse multiple sources of camera position (e.g., GPS on the surface, depth, attitude) and scene structure to yield a robust georeferenced solution.

\subsection{Systematic surveys}
Given the strong attenuation of light underwater \cite{DUNTLEY_1963}, optical imaging is typically performed at distances of around 0.5-3m off the bottom. This limits the footprint of images to be only a few square meters, which implies that extents of tens to thousands of square meters require handling tens to thousands of images. In addition, achieving full coverage is challenging underwater. Systematically covering an area without leaving `holes' and providing adequate overlap has typically accomplished by robots that follow a `mow the lawn' pattern at a near-constant altitude off the bottom \cite{Bingham_2010}\cite{Williams_2010}, insuring image overlap across parallel tracklines. These robots carry a sophisticated navigation suite that typically measures velocity with an acoustic Doppler Velocity Log (DVL), depth through pressure, precise attitude sensing with an inertial measurement unit, and positioning with acoustic transponders, as well as a combination of thrusters and control surfaces to follow a desired survey pattern. Their operating complexity and costs put them beyond many day-to-day scientific applications that would benefit from large scale visual reconstructions.
Diver-held imaging packages can also deliver broad area coverage though the replicating a systematic `mow the lawn' pattern with humans requires ropes as visual guides and additional people in the water to handle the lines \cite{Henderson_2013} 
Our approach for systematic, full coverage underwater photographic surveys is to use a line wound around a drum as a guide. Unwinding the line under tension constrains motion to a spiral pattern. The curve traced by the tip of the line (and the imaging platform) corresponds to the involute of a circle (i.e., the cross section of the drum), with constant separation distance between revolutions corresponding to the perimeter of the circle. A more cumbersome version of this was attempted in 2007 in the expedition reported in \cite{Camilli_2007} with the line coiled on the imaging platform and the diver having to let out the trackline spacing after completing each revolution, resulting in concentric circles. More recently, a `minute mosaic' \cite{gintert2012third} uses a rebar pin as a visual reference for a diver to complete three revolutions with increasing radius while collecting imagery for a 2D mosaic. Since it depends on the diver's assessment of distance to the pin, the areas covered varied from 19 to 44 m^{2}. This is also likely to result in variable image overlap between revolutions, affecting the quality of the composite and potentially limiting its value for systematic repeat surveys. 

Archimidean spirals, that resemble involutes of a circle, have been investigated in the reconstruction of surfaces in metrology \cite{Wieczorowski_2001} and in estimating patchy distributions \cite{Kalikhman_2006} though these uses are concerned with sparse sampling of the area of interest, while our focus is in an operationally simple way to achieve full photographic coverage of an area with limited field of view (and footprint) imaging. Our approach is practical for full coverage of \sim{100m^{2}} areas.

%Bi$_{2}$Sr$_{2}$Ca$_{2}$Cu$_{3}$O$_{10 + \delta}$ (Bi-2223). We conclude with a revisitation of the work of  which can also be found at this URL: \url{http://adsabs.harvard.edu/abs%/1975CMaPh..43..199H}.
  