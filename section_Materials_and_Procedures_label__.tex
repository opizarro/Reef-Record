\section{Materials and Procedures}
\label{sec:MandP}
%Provides a detailed set of instructions for implementing the method, including all materials required and all procedures followed. Describes methods used in evaluation and intercalibration or intercomparison studies. Presents data sources, data extraction protocols and analytical methods used in reviews and metaanalyses.

%Whenever possible and appropriate, manuscripts must present complete instructions for the recommended procedure, analogous to a good cookbook or an easy-to-use laboratory manual. Descriptions of equipment and apparatus must provide a similar level of detail regarding the contruction and operation of the device. Evaluation and intercalibration or intercomparison studies may refer to published descriptions of existing methods, but should describe in reproducible detail how the present study was conducted. Metaanalytical studies should provide details regarding data sources and extraction, and analytical methods used.

For down-looking cameras, systematic surveys covering areas much larger than the footprint of a single image require multiple views of the same scene points (i.e., `image overlap') in order to relate the multiple images into a composite representation such as a 3D reconstruction or an orthographic mosaic. Overlap along the direction of motion depends on the angular field of view, altitude and motion between image capture instants (Fig \ref{fig:fov}). The footprint t is given by $f = 2 h \tan(\beta/2)$, where $\beta$ is the angular field of view and $h$ the altitude (distance from camera to seafloor). The angular field of view can be estimated from a camera calibration (\cite{bouguet2004camera}) or, more approximately, by using the effective focal length in water and the imaging chip size. For example, the configuration that captured the imagery used here has an across track field of view of 42$^\circ$, and 34$^\circ$ along track. At a desired altitude of 2m, the across track footprint is $f_{across}$=1.54~m and the along track footprint is $f_{along}$=1.22~m giving an areal footprint of just under 2m$^2$ per image. 
The footprint of the image together with the displacement between frames $\Delta$ determines the number of views of a scene point $n = f /\Delta$. The along track displacement is $\Delta_{along} = s \cdot T$, where $s$ is the survey speed and $T$ the period between frames. 
%The across track displacement is $\Delta_{across} = 2\pi R$.
For the purposes of designing a survey with a target number of views, we can determine the displacement as $\Delta = f/n$. For along track motion, the survey speed is then
\begin{equation}
s = \frac{f_{along}}{n \cdot T}
\end{equation}
The trackline spacing (across track) is simply $\Delta_{across} = f_{across}/n$.
For example, for the footprints of our configuration, three views of each scene point (n=3) and a period $T$=0.5~s, $s$=0.81~m/s and the trackline spacing should be $\Delta_{across}$=0.51~m

and a minimum of four views of each scene points, we have forward speed X and trackline spacing Y. 

Equation for min number of views here. Solve for speed.
Overlap across tracks depends on the angular field of view, altitude and spacing between tracks. 
Equation for across track overlap. Solve for spacing.

% curves of overlap?  SBW: Worth generating them to see if there is anything interesting in them.
% prediction of overlap and number of links?
