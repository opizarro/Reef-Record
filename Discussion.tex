\section{Discussion}
%You should have already described the results and conclusions of any tests and analyses conducted in the Assessment section.
%For new methods: Authors should discuss the degree to which a new method meets the need defined in the introduction. Authors must address and establish the potential for the method to lead to new insight, based on the demonstrated properties as tested and described in the assessment section. What does this method offer? Is it a fundamentally new approach, or a breakthrough advance in comparison with the capabilities and properties of alternative approaches? What questions and problems might be addressed that were previously intractable? What new questions are raised? Again, you must make plain to the editors and reviewers why you believe that your work will advance the aquatic sciences. Manuscripts that only report modest improvements upon methods already in use are unlikely to be accepted.
%For evaluation, comparison and intercalibration studies, and metaanalyses: Having presented the tests conducted and the conclusions reached, authors should now discuss the impact of their conclusions. How will these conclusions change the interpretation of past work? Have past methodological errors led to any probably-incorrect conclusions? What needs to be changed in future work? Authors must demonstrate that their work will have a substantial impact on the way published work should be interpreted, and on the way future work should be conducted.

%Real-world applications will often require modifications to procedures. Authors are encouraged to conclude with brief comments on particularly critical aspects of the procedure, and suggestions for adapting the method to various potential applications or environments. If an existing method has been re-evaluated, authors should make recommendations for any changes to the method for future work.

% Limitations: operation assumes that the line can 'sweep' unobstructed, which require a relatively flat, though not necessarily horizontal, surface. There's something that can be said about the envelope the spiral can follow in depth as a function of the bathymetric profile under the line. That is, the line has to be taught and has to be able to sweep the terrain between the camera and the pole while, ideally, keeping the camera at the correct imaging height. When following a radial line, this does not allow for decreasing depth (getting shallower) followed by a significant increase in depth as the line would hit the local shallow 'peak'. An improvement could be a drum on low friction sliding 'carriage' along a longer pole, this would allow the drum to stay at the 