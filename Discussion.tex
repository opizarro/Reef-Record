\section{Discussion}
%You should have already described the results and conclusions of any tests and analyses conducted in the Assessment section.
%For new methods: Authors should discuss the degree to which a new method meets the need defined in the introduction. Authors must address and establish the potential for the method to lead to new insight, based on the demonstrated properties as tested and described in the assessment section. What does this method offer? Is it a fundamentally new approach, or a breakthrough advance in comparison with the capabilities and properties of alternative approaches? What questions and problems might be addressed that were previously intractable? What new questions are raised? Again, you must make plain to the editors and reviewers why you believe that your work will advance the aquatic sciences. Manuscripts that only report modest improvements upon methods already in use are unlikely to be accepted.
%For evaluation, comparison and intercalibration studies, and metaanalyses: Having presented the tests conducted and the conclusions reached, authors should now discuss the impact of their conclusions. How will these conclusions change the interpretation of past work? Have past methodological errors led to any probably-incorrect conclusions? What needs to be changed in future work? Authors must demonstrate that their work will have a substantial impact on the way published work should be interpreted, and on the way future work should be conducted.
