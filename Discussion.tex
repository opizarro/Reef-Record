\section{Discussion}
%You should have already described the results and conclusions of any tests and analyses conducted in the Assessment section.
%For new methods: Authors should discuss the degree to which a new method meets the need defined in the introduction.
%Authors must address and establish the potential for the method to lead to new insight, based on the demonstrated properties as tested and described in the assessment section. What does this method offer? Is it a fundamentally new approach, or a breakthrough advance in comparison with the capabilities and properties of alternative approaches? 
%What questions and problems might be addressed that were previously intractable? What new questions are raised? Again, you must make plain to the editors and reviewers why you believe that your work will advance the aquatic sciences. Manuscripts that only report modest improvements upon methods already in use are unlikely to be accepted.
We have shown that the proposed method for constrained motion spiral survey provides a simple yet effective way to systematically survey an area much larger than a single image footprint. The successive passes in the spiral path can be spaced precisely to allow for sufficient overlap across revolutions and enable 3D visual reconstructions from visual SLAM or structure from motion processing. This approach also facilitates georeferencing and precisely repeatable surveys. The particular design we have chosen covers \sim108~m$^2$ and takes approximately 15~minutes to execute. It is robust to swell and wind and takes around 15~minutes per site. With this type of survey data it is straightforward to generate data products such as multi-scale terrain complexity measures \cite{Friedman_2012}.

This method offers field scientists the ability to generate high resolution, broad scale representations of reef environments without depending on engineering specialists and complex robotic systems. It can be integrated into their standard fieldwork for a modest amount of additional effort and enable novel views of structural complexity. When coupled with ecological surveys (e.g., corals and fish), the method can offer valuable data at multiple scales for understanding the relationship between species diversity and habitat complexity  \cite{Graham_2012}. When these surveys are repeated and coupled with environmental data and observations of the physical disturbances, it enables powerful insights into the ecological and evolutionary processes operating in marine systems.

