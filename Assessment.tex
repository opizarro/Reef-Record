\section{Assessment}
%Typically, authors will present in this section the critical experiments or studies that were conducted in the process of methods testing, the results of those studies, and the proof of concept they provide. The Assessment section may also be used to present the results of re-evaluations of existing methods, intercomparison and intercalibration experiments, and metaanalyses. It should include not only the factual results, but their interpretation and the conclusions reached from them. The assessment should provide the answers to such basic questions as:
%How do you know that your method really works?
%How well does your method work?
%What are the method's strengths and limitations?
%How difficult or expensive is your method to adopt and use?
%Does an existing method indeed have a fundamental flaw that needs to be addressed?
%How well did alternative methods agree?
%The methods assessment must address statistical properties of new methods, such as precision, accuracy, and detection limits. These elements are particularly important if a new method is intended to supplant an established procedure.
%If a method involves any subjective decision by an operator, or is dependent on operator skill, the manuscript must address explicitly the extent to which operator performance affects the statistical properties of the method. As an example: epifluorescence microscopy is often used to enumerate aquatic bacteria, but usually requires subjective decisions on the part of the individual doing the counting. The extent to which such subjective decisions influence results would need to be measured.
%Authors should assess the ease or difficulty of setting up and employing the method.
%One effective and persuasive technique to demonstrate the utility of a new method is to apply it successfully to a real-world problem. Authors are not required to demonstrate proof-of-concept through a real-world application, in order to submit a methods manuscript to Limnology and Oceanography: Methods. Demonstrations of the effectiveness of a method under controlled experimental conditions are equally acceptable, provided that the authors can argue successfully that the transition to real-world applications should not present potentially insurmountable obstacles.

%Talk about speed, quality, repeatability
% figures of typical results
We characterize the techniques performance based on three expeditions to Lizard Island, in which over 20 (what's the actual number?) spiral surveys were revisited on each trip.
% repeat survey
\subsection{Operational simplicity and survey speed}
The equipment used is easy to handle. Driving a star picket (in reef) or a surveyor tripod (in general) is a standard task. The swimmer only needs to swim forward and keep a desired altitude over the bottom while keeping tension on the line. 
Using a calibrated stereo camera provides scale information that otherwise requires additional infrastructure and effort (such as deploying markers acting as ground control points and measuring the distance between them).

\subsection{Georeferencing}
For shallow, clear water surveys (~4m or less), while the platform is near the surface it is possible to simultaneously acquire imagery of the bottom and GPS fixes. As long as one image has a GPS fix associated with it and the image overlaps with other images part of the images at survey altitude, the whole reconstruction will be georeferenced. Alternatively, any point of the survey (typically where the star picket is driven into the bottom) can be surveyed in and that location entered manually into the reconstruction.  
The fusion of depth and attitude ensures that the reconstruction is oriented properly and at the current depth.

\subsection{Visual survey quality}
The spiral survey by design allows for constant separation between passes, when matched to the field of view and altitude it guarantees high overlap across successive passes. For our setup this resulted in a network of cameras with X-Y images and Z links on average. Each camera was connected to U other cameras on average and V cameras that are temporaly distant (Figures? Histograms?)

\subsection{Repeatability}
Given the reconstruction from a previous attempt and the GPS coordinate, an experienced diver can relocate the central point in seconds to a few minutes, depending on how much the site has changed.
Figure T presents examples of revisited sites on Lizard Island for 201404, 201410 and 201505.


    
  
  