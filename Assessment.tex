\section{Assessment}
%Typically, authors will present in this section the critical experiments or studies that were conducted in the process of methods testing, the results of those studies, and the proof of concept they provide. The Assessment section may also be used to present the results of re-evaluations of existing methods, intercomparison and intercalibration experiments, and metaanalyses. It should include not only the factual results, but their interpretation and the conclusions reached from them. The assessment should provide the answers to such basic questions as:
%How do you know that your method really works?
%How well does your method work?
%What are the method's strengths and limitations?
%How difficult or expensive is your method to adopt and use?
%Does an existing method indeed have a fundamental flaw that needs to be addressed?
%How well did alternative methods agree?
%The methods assessment must address statistical properties of new methods, such as precision, accuracy, and detection limits. These elements are particularly important if a new method is intended to supplant an established procedure.
%If a method involves any subjective decision by an operator, or is dependent on operator skill, the manuscript must address explicitly the extent to which operator performance affects the statistical properties of the method. As an example: epifluorescence microscopy is often used to enumerate aquatic bacteria, but usually requires subjective decisions on the part of the individual doing the counting. The extent to which such subjective decisions influence results would need to be measured.
%Authors should assess the ease or difficulty of setting up and employing the method.
%One effective and persuasive technique to demonstrate the utility of a new method is to apply it successfully to a real-world problem. Authors are not required to demonstrate proof-of-concept through a real-world application, in order to submit a methods manuscript to Limnology and Oceanography: Methods. Demonstrations of the effectiveness of a method under controlled experimental conditions are equally acceptable, provided that the authors can argue successfully that the transition to real-world applications should not present potentially insurmountable obstacles.
This is where ...