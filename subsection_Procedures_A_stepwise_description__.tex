\subsection{Procedures}
%A stepwise description of all procedures used, divided into sub-procedures as necessary for clarity. Sufficient detail must be provided that a reader may readily adopt and employ the method as described, at least under the conditions described.
%Limnology and Oceanography: Methods will not accept manuscripts that describe laboratory and field techniques, equipment, analyses, and other methods, in insufficient detail to be reproduced by others.

In the field using two people: a swimmer and an assistant:\begin{enumerate}
\item Select location, such that the spiral covers the area of interest.
\item Drive star picket at the center of the area of interest. 
\item Attach pole and drum to star picket.
\item Clip imaging package to line.
\item The swimmer pushes the imaging package forward while keeping tension on the line and the desired altitude. Continuing until the line has completely unwound from the drum. In case of rough conditions, the assistant stays by the pole tending the line.
\item Once the survey is completed, the swimmer detaches the line from the imaging platform. The assistant coils the line and takes the pole and drum off the star picket. 
\item The swimmer then goes with the imaging platform back over the center of the survey, completing at least four `spokes'.  
\end{enumerate}

The imagery and navigation logs are post processed using the ACFR pipeline \cite{Johnson_Roberson_2010} \cite{Mahon_2008} \cite{Johnson_Roberson_2013} to estimate camera poses and 3D composite meshes of the surveyed area. Multi-scale structural complexity indeces such as rugosity are then derived from those meshes \cite{Friedman_2012} 
