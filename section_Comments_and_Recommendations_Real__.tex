\section{Comments and Recommendations}

%Real-world applications will often require modifications to procedures. Authors are encouraged to conclude with brief comments on particularly critical aspects of the procedure, and suggestions for adapting the method to various potential applications or environments. If an existing method has been re-evaluated, authors should make recommendations for any changes to the method for future work.

One of the limitations of the technique as presented is that it assumes that the line between the imaging platform and drum can 'sweep' unobstructed, which requires a relatively flat, though not necessarily horizontal, surface. There's something that can be said about the envelope the spiral can follow in depth as a function of the bathymetric profile under the line. That is, the line has to be taught and has to be able to sweep the terrain between the camera and the pole while, ideally, keeping the camera at the correct imaging height. When following a radial line, this does not allow for decreasing depth (getting shallower) followed by a significant increase in depth as the line would hit the local shallow 'peak'. An improvement could be a drum on low friction sliding 'carriage' along a longer pole, this would allow the drum to stay at the 