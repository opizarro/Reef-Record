\section{Comments and Recommendations}

%Real-world applications will often require modifications to procedures. Authors are encouraged to conclude with brief comments on particularly critical aspects of the procedure, and suggestions for adapting the method to various potential applications or environments. If an existing method has been re-evaluated, authors should make recommendations for any changes to the method for future work.

One of the limitations of the technique is that the line between the imaging platform and drum must be free to 'sweep' the site unobstructed. This is satisfied by a relatively flat, though not necessarily horizontal, surface. In practice, the technique works best for terrain that can be approximated as a plane or a local maximum or minimum, with height variations that can be absorbed by an imaging configuration that can remain in focus for the expected variations in altitude, and an image footprint that still achieves overlap with neighboring revolutions at the low end of the allowable altitude range.
More precisely, the during a survey the line has to be taught and has to be able to sweep above the terrain between the camera and the pole while, ideally, keeping the camera at the correct imaging height. When following a radial line, this does not allow for decreasing depth (getting shallower) followed by a significant increase in depth further out as the line would hit the local shallow 'peak'. This is particularly problematic if the shallow point is shallower than the depth of the drum since it imposes a positive slope to the line and potentially an altitude much higher than desired if the terrain drops off after the shallow peak. This problem can be avoided in some types of terrain by placing the center point at the local maximum or minimum of the patch to be surveyed. It also it tends to be less significant if the patch size is kept small such that the local terrain can be approximated by a plane or local maximum or minimum.
An improvement to the equipment would be a drum on low friction sliding 'carriage' along a longer pole, this would allow the drum to change its depth, keeping the line horizontal rather than imposing a slope detrimental to imaging past shallow peaks.

This technique has been mostly used on carbonate reefs, where a temporary or permanent star picket can be driven into the substrate and then serve as an attachment point for the pole. In cases of rocky reefs or soft sediments, different attachment methods are required. An alternative would be to use a pole with a heavy base or tripod. The increase in versatility of bottom types on which the technique comes at the price of a more awkward transport in water of the equipment to the site. 