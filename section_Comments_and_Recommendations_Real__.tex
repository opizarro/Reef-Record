\section{Comments and Recommendations}

%Real-world applications will often require modifications to procedures. Authors are encouraged to conclude with brief comments on particularly critical aspects of the procedure, and suggestions for adapting the method to various potential applications or environments. If an existing method has been re-evaluated, authors should make recommendations for any changes to the method for future work.

One of the limitations of the technique is that the line between the imaging platform and drum must be free to 'sweep' the site unobstructed. This is satisfied by a relatively flat, though not necessarily horizontal, surface. In practice, the technique works best for terrain that can be approximated as a plane or a local maximum or minimum, with height variations that can be absorbed by an imaging configuration that can remain in focus for the expected variations in altitude, and an image footprint that still achieves overlap with neighboring revolutions at the low end of the allowable altitude range.

This technique has been mostly used on carbonate reefs, where a temporary or permanent star picket can be driven into the substrate and then serve as an attachment point for the pole. In cases of rocky reefs or soft sediments, different attachment methods are required. An alternative would be to use a pole with a heavy base or tripod. The increase in versatility of bottom types on which the technique comes at the price of a more awkward transport in water of the equipment to the site. 