
\subsection{Repeatability for monitoring purposes}
Given a printout of the mosaic from a previous survey and the georeferenced location of the center of the reef record, an experienced swimmer can relocate the central point in seconds to a few minutes, depending on how much the site has changed. The typical procedure was to approach the site by boat using a hand-held GPS.  Once within a range of a few meters, the pole was dropped near the desired location and the swimmer used the platform's display to guide them to the GPS coordinates of the center of the survey area.
If the particular application allows the star picket to be left embedded in the substrate, relocating the survey site is trivial. This approach will be robust to substantial changes in appearance that can occur after events such as large storms.  If the star picket could not be left, a printout of the site map on waterproof paper was used to help locate the center of the survey area relative to the features evident in the map.
Figure \ref{fig:sitemap} shows locations of sites revisited on Lizard Island for April 2014, October 2014, May 2015 and November 2015.
