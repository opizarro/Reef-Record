
\subsection{Materials}
%"Materials" includes expendible and non-expendible supplies, equipment, and solutions. Materials should be listed as completely as possible, with careful attention to providing enough information to assist new users in adopting the method. Descriptions of needed equipment must note the essential features of the equipment in sufficient detail to allow users to obtain similar devices, or authors may refer to specific commercial products in lieu of a detailed description. The composition of any solutions must be stated explicitly (e.g. grams per liter of a specific compound).
We rely on three three major components for the data acquisition
\begin{enumerate}
\item Camera system. We use an updated version of the instrumented stereo pair used in \cite{Henderson_2013} \cite{Camilli_2007}, integrated into a smaller form factor and including an acoustic altimeter. We require a calibrated stereo pair and sensor suite, as well as the pose offset of each sensor relative to the navigation center \cite{Johnson_Roberson_2013} \cite{Mahon_2008}. A single camera can be used with appropriate external references and processing tools (see Section \ref{sec:VisRec}).
\item drum and line. This diameter of the drum is determined by the desired spacing between successive revolutions. This, in turn should be informed by the desired overlap given the image footprint at the target altitude. The length of the line determines the overall radius of the patch being surveyed. The resulting pattern will have the form shown in Figure \ref{fig:spiral}. The drum is attached to a pole of length close to the target altitude, typically 1.5-2.0 m. See Figure \ref{fig:pole}.

% photo of drum and line

% full equation and approximate equation for drum and line length


\item star picket or base. For reef structures, a star picket driven into the substrate serves as the anchor point to hold the drum and pole, as well as the center of the survey patch.
% photo of base / star picket

\end{enumerate}


  